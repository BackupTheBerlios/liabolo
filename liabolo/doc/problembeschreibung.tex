
Der uns erteilte Auftrag besteht in der Entwicklung einer XML-Datenbank\index{XML-Datenbank}, die vornehmlich zur 
Verwaltung eines Buchbestandes
eingesetzt werden soll. Im Unternehmen unseres Auftraggebers sind 10 Personen besch\"aftigt, es m\"ussen ca. 8000
Einheiten von der Datenbank verwaltet werden. Weiterhin w\"are eine Anbindung an die \"ortliche Universit\"atsbibliothek
w\"unschenswert. Bisher wurde Liman Pro eingesetzt; ein Mitarbeiter verwaltet seinen eigenen Literaturbestand durch Microsoft
Excel, ein Import von Exceldokumenten wird angestrebt. 60 Prozent des gesamten Bestandes sind bisher erfasst; wie die 
Umstellung auf das neue System erfolgen soll, ist noch nicht im Detail er\"ortert worden. 

Die Datenbank soll auf mehreren Rechnerplattformen einsetzbar sein, insbesondere seien hier Linux\index{Linux} und 
Microsoft Windows\index{Microsoft}
zu nennen. Die Datenbank wird auf einer 'Client-Server'-Architektur\index{Client-Server Architektur} basieren, wobei 
ein Offline-Betrieb auf einem 
Client-Rechner und anschlie{\ss}ende Synchronisation\index{Synchronisation} mit dem Server m\"oglich sein soll. Desweiteren 
ist hohe Bedienbarkeit
gefordert, sowie eine Sicherung des administrativen Zugriffs auf den Datenbestand durch Passw\"orter\index{Passwort}. Da dem 
Auftraggeber
sehr daran gelegen ist, seinen Literaturbestand auch f\"ur au{\ss}erbetriebliche Personen sichtbar zu machen, soll eine
Schnittstelle zum World Wide Web \index{World Wide Web (WWW)} zur Verf\"ugung gestellt werden.

In der Vergangenheit traten gelegentlich Probleme beim Drucken mit einem \index{Postscript} Postscript-Drucker \index{Drucker}
 auf. Ein letzter Kritikpunkt
am bisherigen System ist die Beschr\"ankung, dass nur Zeitschriften \index{Zeitschrift} und B\"ucher \index{Buch} erfasst werden 
k\"onnen; die neue Datenbank soll also genereller einsetzbar sein.