\section{Signaturbildung}%Thorsten  
\index{Signaturbildung}
\subsection{Signaturbildung im Bibliothekswesen}

Im Bibliothekswesen bezeichnet die Signatur \index{Signatur} den Standort eines Buches \index{Buch} in der \index{Bibliothek} 
Bibliothek \cite{signatur_begriff}.
Eine solche Signatur ist eineindeutig: es gibt f\"ur \textit{jedes} Exemplar eines Buches eine
eindeutige Signatur, selbst wenn von dem Buch mehrere identische Kopien existieren.

Zur Bildung einer solchen Signatur gibt es verschiedene Ans\"atze; an dieser Stelle
sollen zwei grob beschrieben werden: die Signaturbildung anhand der \textit{Bremer Systematik} \index{Bremer Systematik}und der 
\textit{Regensburger Verbundklassifikation}. \index{Regensburger Verbundklassifikation}

\subsubsection{Bremer Systematik}

Die Bremer Systematik oder auch Bremer Fachsystematik kommt zum einen (nat\"urlich) an
der Staats- und Universit\"atsbibliothek Bremen, aber auch an der hiesigen Bibliothek der
Carl von Ossietzky Universit\"at Oldenburg zum Einsatz. Anhand der Fachsystematik wird
die Signatur gebildet, indem ein K\"urzel des Fachs als zentrales Element in die Signatur
\"ubernommen wird. Die Bremer Fachsystematik ist online zu finden unter \cite{fachsystematik}.
An der Bibliothek der Universit\"at Oldenburg wird die Signatur mit Hilfe der Systematik wie
folgt gebildet:

\begin{itemize}

\item drei Buchstaben, die das Fach angeben (``Fachzuweisung''),
\item einer Zahlenkombination, die die Systemstelle (bzw. das Fachgebiet)
    innerhalb des Faches angibt (``systematische Notation'') und
\item einer ``Aufstellungsnummer'', bestehend aus zwei Buchstaben und vier Ziffern.

\end{itemize}

\noindent \textbf{Beispiel:} Bsp: inf 724 CK 5213 



\subsubsection{Regensburger Verbundklassifikation}

Als Studenten der Carl von Ossietzky Universit\"at war es f\"ur uns naheliegend, die Bremer Systematik
zu verwenden. Dennoch wurde im Vorfeld auch die Signaturbildung anhand der Regensburger Verbundklassifikation 
betrachtet, die der Vollst\"andigkeit halber hier kurz beschrieben wird. 
Eine Online-Version der Klassifikation findet sich unter \cite{rvk}.
Die Signatur besteht aus \cite{sig_regensburg}:
\begin{itemize}
\item einem Lokalkennzeichen (2-3 Ziffern), gefolgt von einem Schr\"agstrich ('/'),
\item einer Systemstelle (2 Grobuchstaben + mehrstellige Zahl), die sich anhand der Klassifikation ergibt
\item und einem individualisierenden Element (Cluster-Sanborn-Notation bzw. 
   Erscheinungsjahr, ggf. noch Auflage, Band, Exemplar).
\end{itemize}

\noindent \textbf{Beispiel:} 17/GE 4001 B724 (9) -2 +3

\subsection{Signaturbildung im Projekt}

\subsubsection{Allgemeines}

Als Signatur zur Kennzeichnung der verschiedenen Exemplare des
Bibliotheksbestand orientieren wir uns an der sogenannten "'Bremer Systematik"',
indem wir die von ihr verwendeten Fachk\"urzel (bzw. Fachzuweisungen) 
als Voreinstellung f\"ur die Kategorien \index{Kategorie}anbieten.
Die von uns verwendete Signatur setzt sich wie folgt zusammen:

\begin{itemize}

    \item drei Buchstaben, die das Fach angeben ("'Fachzuweisung"'),

    \item gefolgt von einem Unterstrich ("'\_"'),
    
    \item und einer laufenden Nummer von 1 bis 999999999, die hexadezimal dargestellt wird, also
      bis 2540BE3FF. Durch die hexadezimale Darstellung soll die Lesbarkeit der Nummer 
      erh\"oht werden.
    
\end{itemize}




\subsubsection{Algorithmus zur Signaturbildung} \index{Signaturbildung}
    
    \begin{enumerate}
        
        
          
        \item Bestimme das Fachgebiet \index{Fachgebiet} anhand der Bremer Systematik (oder sonstiger zur
          Verf\"ugung stehenden Kategorien) in Form eines
          K\"urzels aus drei Kleinbuchstaben

        \item H\"ange einen Unterstrich an das K\"urzel an
        
        
        \item Bestimme die fortlaufende Nummer des Gesamtbestandes wie folgt:
          \begin{itemize}
            
          \item Die erste Nummer des Bestands ist 000000001.
          
          \item Die letzte verwendete Nummer (die dem System bekannt sein muss) wird um
              eins erh\"oht und an die bisherige Signatur angef\"ugt.
          
          \end{itemize}

   \end{enumerate}
\textit {Hinweis:} Konsequenz des obigen Algorithmus ist, dass jedes Objekt der Datenbank 
eine eigene Signatur erh\"alt, also z.B. B\"ucher, die mit mehreren Exemplaren im Bestand
vorhanden sind, besitzen jeweils eine individuelle Signatur.

\subsection{URI, URN}

\subsubsection{Was sind URIs, URLs und URNs?}

URI \index{URI} ist die Abk\"urzung f\"ur \textit{Uniform Resource Identifier}. Ein URI ist eine kompakte
Zeichenkette zur Identifizierung einer abstrakten oder physikalischen Ressource \cite{rfc2396}.
Eine Ressource \index{Ressource} kann alles sein, was einen Namen hat oder beschrieben werden kann, z.B. Dokumente, 
Bilder oder ein Service.
Weiterhin versteht man unter URI den Oberbegriff f\"ur die Unterarten \textit{Uniform Resource Locator}(URL)
und \textit{Uniform Resource Name}(URN). 

URLs \index{URI} identifizieren eine Ressource \"uber den Ort (im WWW), an dem sie zu
finden ist, nicht \"uber einen Namen oder Attribute \cite{rfc2396}. Sie stellen also 
gewisserma{\ss}en einen Zeiger auf die Ressource dar.

URNs \index{URN} sollen als best\"andige, ortsunabh\"angige Bezeichner f\"ur Ressourcen dienen \cite{rfc2141}. Eine URN
bezeichnet eine Ressource eindeutig, d.h. ist sie mehrfach im Netz vertreten, besitzt sie immer
denselben URN, wo hingegen sie mehrere unterschiedliche URLs hat.

\subsubsection{Syntax}

\paragraph{Uniform Resource Identifier}

Die komplette Syntax von URIs ist im \index{Request for Comment} 'Request for Comment' (RFC) 2396 \cite{rfc2396} angegeben. 
An dieser Stelle wird
nur ansatzweise auf einige wichtige Merkmale eingegangen.

\begin{itemize}

\item Grunds\"atzlich sind nur Zeichen des US-ASCII Zeichensatzes erlaubt. Davon abweichende Zeichen k\"onnen
  mit Hilfe des ``Escape''-Zeichens '\%' benutzt werden.

\item Jeder URI beginnt mit dem Namen eines Schemas, bestehend aus einer Zeichenfolge, die mit ':' abschliesst, 
z.B. ``http:'' oder ``ftp:''.

\item Ein URI ist wie folgt aufgebaut (Auszug):

  \begin{verbatim}
    URI       = scheme ":" hier_part [ "?" query] 
                ["#" fragment]
    
    hier_part = net_path | abs_path | rel_path
    
    net_path  = "//" authority [ abs_path ]
    abs_path  = "/" path_segments
    rel_path  = rel_segments [ abs_path ]

    ...

  \end{verbatim}

\end{itemize}

\noindent \textbf{Beispiel:} $\underbrace{foo}_{scheme}:\underbrace{//example.com:8042}_{authority}
\underbrace{/over/there}_{path}?\underbrace{name=ferret}_{query}\#\underbrace{nose}_{fragment}$


\paragraph{Uniform Resource Name}

Alle URNs haben die gleiche Syntax \cite{rfc2141}:

\begin{verbatim}

  <URN> ::= "urn:" <NID> ":" <NSS>

\end{verbatim}

NID bedeutet \textit{Namespace Identifier}. Ein Namespace (Namensraum) ist eine Menge von einheitlichen
Bezeichnern. Ein NID beginnt mit einem Buchstaben oder einer Zahl gefolgt von einer Folge aus 
Buchstaben, Zahlen und dem '-' Zeichen. Gro{\ss} und Kleinschreibung wird nicht unterschieden.

NSS bedeutet \textit{Namespace Specific String}. Er besteht aus Buchstaben, Zahlen, reservierten
Zeichen ('\%', '/', '?', '\#') und weiteren Sonderzeichen\footnote{f\"ur Details siehe \cite{rfc2141}}.
Bei den reservierten Zeichen ist zur Zeit nur '\%' gebr\"auchlich, die anderen wurden reserviert f\"ur
eine zuk\"unftige Nutzung, sollen aber im Moment nicht benutzt werden.

\noindent \textbf{Beispiel:} $\underbrace{urn}_{scheme}:\underbrace{example:animal:ferret:nose}_{path}$


\subsection{Kategorisierung von Dokumenten}

Eine Kategorie \index{Kategorie} besteht in \textit{Tooliban} aus einem (die Kategorie beschreibenden) Wort und einer
aus diesem Wort abgeleiteten Abk\"urzung, die aus drei Kleinbuchstaben besteht und in die Signatur \index{Signatur}
von Objekten \index{Objekt} aufgenommen wird, die dieser Kategorie zugeordnet werden. \textit{Tooliban} enth\"alt 
eine vorgefertigte Liste von Kategorien. Diese Kategorien entsprechen den F\"achern und Fachk\"urzeln 
der \index{Bremer Systematik} 'Bremer Systematik'. Zus\"atzlich gibt es noch eine Kategorie 'not available' f\"ur alle
Objekte, die keiner Kategorie zugeordnet werden k\"onnen.

Des Weitern ist es Personen mit administrativem Zugang zur Datenbank gestattet, weitere Kategorien 
hinzuzuf\"ugen. Das Entfernen von Kategorien ist allerdings nur erlaubt, wenn noch kein Objekt (z.B. Buch)
innerhalb dieser Kategorie erstellt wurde. Der Grund daf\"ur ist, dass nach der Erstellung eines 
Eintrages eine Signatur gebildet wird, die das Kategorienk\"urzel beinhaltet. Um die Eineindeutigkeit
der Signatur zu gew\"ahrleisten, ist ein Entfernen der Kategorie nicht mehr m\"oglich.

