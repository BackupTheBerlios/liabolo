\chapter*{Einf�hrung}

Seit dem explosionsartigen Wachstum des Internet ist Informationsbeschaffung nicht mehr das Hauptproblem bei der Arbeit. Vielmehr muss das relevante Wissen
aus der Informationsflut herausgefiltert werden. Die so gewonnenen Erkenntnisse m�ssen dann noch effizient verwaltet werden, damit ihr Wert noch gesteigert
werden kann. Waren fr�her Informationen haupts�chlich in B�chern vorhanden, so sammelt der Mensch heute zus�tzlich Hefte, Artikel, sowie Bild- und Tondokumente. F�r diese Sammlung  wird eine geeignete Software ben�tigt, in der die vorhandenen Dokumente erfasst und kategorisiert werden k�nnen. Durch die Mobilit�t
der modernen Gesellschaft ist es mitunter sogar hilfreich, wenn der Aufbewahrungsort f�r ein Dokument elektronisch erfasst werden kann. Auch das eigene
Publizieren w�rde automatisch unterst�tzt. Wenn alle Dokumente bereits elektronisch erfasst sich, erstellt sich eine Quellenangabe fast von alleine. Und da
man nicht alle B�cher selbst kaufen kann, ist es notwendig, auch Dokumente einbinden zu k�nnen, die von anderen bereit gestellt werden. 

\section*{Ziele}
Aus den oben genannten Beispielen ergeben sich die Ziele f�r das Projekt. Alle vorhandenen Dokumente sollen datenbankgest�tzt mit einer grafischen
Benutzeroberfl�che erfasst werden k�nnen. Dabei sollen sowohl Text-, als auch Bild- und Tondokumente unterst�tzt werden. Die Software muss plattformunabh�ngig zur Verf�gung stehen, um unter Windows, Linux und MacOS lauff�hig zu sein. Weiterhin sollen Daten auch mit anderen Nutzern ausgetauscht werden k�nnen, wobei mobile Nutzer auch ohne Internetverbindung mit der lokalen Datenbank arbeiten k�nnen sollen.
\paragraph{}
Das Backend verwaltet die Metadaten in einer nativen XML-Datenbank, hierbei werden Schnittstellen f�r den Online- und Offlineclient bereitgestellt.

 \paragraph{}
 Es soll ein Frontend in Form eines Offline- und Online-Clients entworfen und implementiert werden. Die Clients sollen  eine grafische Benutzungsschnittstelle zum Benutzer bereitstellen und eine komfortable und intuitive Verwaltung der Metadaten erm�glichen.

%       \section*{Forschungsfragen}
%       \section*{Methodik}



