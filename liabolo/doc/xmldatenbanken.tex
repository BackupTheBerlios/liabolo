				\section{XML-Datenbanken}%Stefan
					Laut Pflichtenheft sollte die Datenspeicherung in einer XML-Datenbank erfolgen, da es sich bei den Daten grunds�tzlich nur um XML-Daten handelt. Eine weitere Einschr�nkung war die Voraussetzung, da� es sich um eine Open-Source-L�sung handelt. Die folgenden beiden Vertreter von XML-Datenbanken wurden zum Vergleich ausgew�hlt. Beide Alternativen stellen eine Implementierung der XML:DB-API \index{XML:DB-API} \cite{xmldb_api} bereit. Eine �bersichtliche Aufstellung aller weiteren verbreiteten XML-Datenbanken wurde von Ronald Bourret angefertigt \cite{xml_database_comparison}. 
					\subsection{Xindice}
						Xindice\index{Xindice} (gesprochen [zeen-dee-chay]) wird von der Apache Software Foundation\index{Apache Software Foundation} entwickelt und stellt eine Weiterentwicklung von \textit{dbXML 	1.0}\index{dbXML} dar. Die Anbindung an verschiedene Programmiersprachen ist durch ein XML-RPC Plugin\index{XML-RPC Plugin} gew�hrleitest. Speziell f�r Java ist eine XML:DB API Implementierung entwickelt worden. Zur Zeit werden XPATH\index{XPATH} als Anfragesprache und XML:DB-XUpdate\index{XML:DB-XUpdate} als �nderungssprache verwendet.\\
						Folgende Eigenschaften zeichnet Xindice aus :\\
						 \begin{itemize}
						 	\item{Strukturierung durch Collections : Dokumente werden in Collections gespeichert(entsprechen Verzeichnissen in einem Betriebssystem)}
						 	\item{XPATH Such-Engine : Zur Suche wird XPATH verwendet, das vom W3C(\url{http://www.w3.org/})\index{W3C} spezifiziert wurde. XPATH ist ein bew�hrter Anfrage-Standard auf XML-Dokumenten.}
						 	\item{Indezierung :  Zur Verbesserung der Suchperformance k�nnen Dokumentelemente etc. indexiert werden.}
						 	\item{XML:DB XUpdate Implementierung : M�glichkeit der serverseitgen Daten�nderung.}
						 	\item{XML:DB API Implementierung : Zur Verbundenheit von Java und XML wurde eine Java-Schnittstelle implementiert.}
						 	\item{Kommandozeilen-Befehle : Nahezu alle Funktionen der XML:DB API k�nnen auch durch Eingabe von Kommandos erfolgen.}
						 	\item{Modulare Architektur : Modularer Aufbau von Xindice zur einfachen Erweiterung und Integration in bestehenden Systeme.}
						 \end{itemize}
						F�r n�here Informationene siehe :\cite{xindice},\cite{xindice_admin_guide},\cite{xindice_user_guide},\cite{xindice_developer_guide},\cite{xindice_tools}.
					\subsection{eXist}
						eXist ist eine Open-Source\index{Open-Source} native XML-Datenbank (zur Zeit Version 1.0), die von Wolfgang M. Meyer entwickelt
						wurde. Wie bei der vorgestellten Alternative Xindice ist auch hier die M�glichkeit der Einbettung in anderer Applikationen, in
						eine Servlet-Engine sowie ein Stand-alone-Betrieb vorgesehen. F�r die Einbettung in einen Servlet-Container wird
						Jetty(\url{jetty.mortbay.org}) verwendet, wobei andere Servlet-Container ebenfalls eingesetzt werden k�nnen. eXist zeichnet sich
						durch folgende Eigenschaften aus:\\
						\begin{itemize}
							\item{Automatische Index-Erstellung : Index-Erstellung, um Performance zu steigern}
							\item{XPath/XQuery Such-Engine : Neben XPATH sind auch XQuery-Anfragen m�glich}
							\item{Volltextsuche : Mittels Indexierung performant gel�st}
							\item{XML:DB XUpdate : M�glichkeit um Daten serverseitig zu �ndern}
							\item{Zugriff �ber XML:DB API, HTTP, XML-RPC, SOAP, WebDAV : Verschiedenste Netzwerkanbindungen}
							\item{Backup/Restore Funktion : Backup der kompletten Datenbank mit allen Zugriffsbeschr�nkungen etc.}
							\item{Authorisierungsmechanismus : Ein an UNIX-Authorisierung angelehntes Verfahren(Jedoch nicht im XML:DB API Standard enthalten)}
						\end{itemize}
						F�r n�here Informationene siehe: \cite{xindice},\cite{exist_doc},\cite{exist_doc2},\cite{xml_data_management},\cite{xml_database_comparison}.						
					\subsection{Fazit}
						Der recht kurze �berblick der beiden vorgestellten XML-Datenbanken zeigt grunds�tzliche �bereinstimmungen sowie spezielle
						Unterscheidungen. Beide L�sungen sind native XML-Datenbanken, k�nnen leicht integriert werden und sind �ber die XML:DB API
						seitens Java ansteuerbar. eXist ist zus�tzlich in der Lage, automatisch einen Index zu erstellen und darauf Volltextsuchen
						durchzuf�hren. Weiterhin ist ein Benutzermanagement sowie eine Backup/Restore Funktionalit�t gegeben, die f�r das angehende
						Projekt unerl��lich sind. Insgesamt scheint eXist mehrere Features zu haben, die eventuell im Verlaufe des Projektes noch genutzt
						werden k�nnen. Ein Performance-Gegen�berstellung spricht ebenfalls f�r eXist. Diese Gegen�berstellung ist unter \cite{exist_doc2} zu finden.\\ 
						Insgesamt fiel die Entscheidung also auf die Open-Source L�sung 'eXist', die nun f�r die Datenhaltung sorgen wird.
