\chapter{Erkl�rung spezieller Dokumente}
\label{appendix}
Die folgenden Listings zeigen den Aufbau von Basisdokumente in Liabolo.

\section{Standorte}
	Standorte werden in Liabolo getrennt verwaltet und in den jeweiligen Medientypen referenziert, wobei die Referenzierung manuell aufgel�st wird und somit nciht konsistenzerhaltend ist. Daher ist in Liabolo1.0 die M�glichkeit des Entfernens von Standorten unterdr�ckt.\\
	
	Standorte werden durch einen Namen und zugeh�rige Beschreibung in Listing (\ref{lst:locationsDesc})spezifiziert.
	\lstinputlisting[style=xml,title=Beschreibung der Standorte, label=lst:locationsDesc,caption=Beschreibung derStandorte]{listings/locations.xml}

\section{Individuallisten}
	Individuallisten erm�glichen dem Benutzer eine �bersicht und schnellen Zugriff auf thematisch eng verwandte Dokumente, wie z.B. die Literatur, die f�r eine Diplomarbeit verwendet wird. Listing (ref{lst:indlistDesc}) besitzt einen Namen mit zugeh�riger Beschreibung und f�r jeden Medieneintrag ebenfalls neben dem Namen eine Beschreibung(z.B. wof�r die Literatur eingesetzt wird).
	\lstinputlisting[style=xml,title=Aufbau einer Individualliste, label=lst:indlistDesc,caption=Aufbau einer Individualliste]{listings/indlist.xml}

\section{Medientypen}
F�r jeden Medientyp, wird in der Datenbank ein XML-Dokument mit dem Namen des Medientypen angelegt. Ein Medientyp enth�lt eine Anzahl an Metadaten, die aus der Menge aller verf�gbaren Metadaten, die Liabolo bereitstellt ausgew�hlt werden kann. Jeder Metadaten-Eintrag verf�gt �ber die Attribute\\
\begin{itemize}
	\item{name: Der benutzerdefinierte Name des Metadateneintrages}
	\item{desc: Die zugeh�rige Beschreibung}
	\item{DCid: Das Mapping auf einen Basis-Metadaten-Typ, welche von Liabolo vorgegeben werden. Damit soll die M�glichkeit offen gehalten werden, da� z.B. mehrere Autoren eines Buches angezeigt werden, sie sich jeddoch immer auf den Basistyp 'author' beziehen}
	\item{type: Gibt an, ob es sich um eine Textzeile, einen Textbereich, etc. handelt. Wird zur Darstellung in der Gui ben�tigt.}
\end{itemize}
mit denen ein Eintrag n�her spezifiziert werden kann.\\
Listing (\ref{lst:bookDesc}) zeigt die Struktur eines Buches auf.
	\lstinputlisting[style=xml,title=Aufbau einer Buchbeschreibung, label=lst:bookDesc,caption=Aufbau einer Buchbeschreibung]{listings/book.xml}

\newpage
\section{Branches}
Das Listing(\ref{lst:branchesDesc}) zeigt alle verf�gbaren branches mit deren Beschreibung auf.
	\lstinputlisting[style=xml,title=Beschreibung der verf�gbaren Branches, label=lst:branchesDesc,caption=Beschreibung der verf�gbaren Branches, firstline=1, lastline=20 ]{listings/branches.xml}